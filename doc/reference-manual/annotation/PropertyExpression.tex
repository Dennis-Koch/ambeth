\SetAPI{J}
\section{PropertyExpression}
\label{annotation:PropertyExpression}
\ClearAPI

The annotation \javadoc{de.osthus.ambeth.expr.PropertyExpression}{PropertyExpression} allows the usage of expressions, for example ( a + b ) to be used on properties of entities. 
\begin{lstlisting}[style=Java,caption={Example usage for ``just calculation''}]

interface SomeEntity {

	@PropertyExpression("3+5")
	String getCalculatedValue();

}

\end{lstlisting}
This would just return always ``8''. 
Ambeth now allows the usage of variables inside an expression.

\begin{lstlisting}[style=Java,caption={Example usage for ``calculations with variables''}]

interface SomeEntity {

	@PropertyExpression(" 3 * ${MyVariable}")
	String getCalculatedValue();
	
	String getMyVariable();

}

\end{lstlisting}
\ifx{\verb+$+}!\fi % stop the green of the $ sign
Here an variable of that entity is used. It also possible to use multiple variable resolution.

\begin{lstlisting}[style=Java,caption={Example usage for ``calculations with variables''}]

interface SomeEntity {

	@PropertyExpression(" 3 * ${MyVariable${MyLetter}}")
	String getCalculatedValue();
	
	String getMyVariableA();
	
	String getMyVariableB();
	
	String getMyLetter();

}

\end{lstlisting}
Here the variable ``MyLetter'' is used to decide between the two variables.

\subsection{Built-in operators}

\begin{itemize}
	\item Addition: 2 + 2
	\item Subtraction: 2 - 2
	\item Multiplication: 2 * 2
	\item Division: 2 / 2
	\item Exponentation: 2 \textasciicircum 2
	\item Unary Minus,Plus (Sign Operators): +2 - (-2)
	\item Modulo: 2 \% 2
\end{itemize}

\subsection{Built-in functions}

\begin{itemize}
	\item abs: absolute value
	\item acos: arc cosine
	\item asin: arc sine
	\item atan: arc tangent
	\item cbrt: cubic root
	\item ceil: nearest upper integer
	\item cos: cosine
	\item cosh: hyperbolic cosine
	\item exp: euler's number raised to the power (e\textasciicircum x)
	\item floor: nearest lower integer
	\item log: logarithmus naturalis (base e)
	\item log10: logarithm (base 10)
	\item log2: logarithm (base 2)
	\item sin: sine
	\item sinh: hyperbolic sine
	\item sqrt: square root
	\item tan: tangent
	\item tanh: hyperbolic tangent
\end{itemize}

\subsection{Scientific notation}
The number is split into a significant/mantissa y and exponent x of the form yEx which is evaluated as y * 10 \textasciicircum x. Be aware that 'e/E' is no operator but part of a number and therefore an expression like 1.1e - (x*2) can not be evaluated. An example using the Fine-structure constant \alpha = 	7.2973525698 * 10\textasciicircum -3: ``7.2973525698e-3''

\subsection{Implicit multiplication}
The expression API does support implicit multiplication. Therefore an expression like 2cos(yx) will be interpreted as 2*cos(y*x)


%% GENERATED LISTINGS - DO NOT EDIT
\inputjava{Annotation definition \type{PropertyExpression}}
{jambeth-expr/src/main/java/de/osthus/ambeth/expr/PropertyExpression.java}
%% GENERATED LISTINGS END