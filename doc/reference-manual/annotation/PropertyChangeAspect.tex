\SetAPI{J}
\section{PropertyChangeAspect}
\label{annotation:PropertyChangeAspect}
\ClearAPI
\javadoc{com.koch.ambeth.annotation.PropertyChangeAspect}{PropertyChangeAspect}
%% GENERATED LISTINGS - DO NOT EDIT
\inputjava{Annotation definition \type{PropertyChangeAspect}}
{jambeth-util/src/main/java/com/koch/ambeth/util/annotation/PropertyChangeAspect.java}
%% GENERATED LISTINGS END
Can be used to annotate bean declarations and because of this have them enhanced with PCE (property change event) capabilities at runtime. The annotation is processed by the \javadoc{com.koch.ambeth.propertychange.PropertyChangeInstantiationProcessor}{PropertyChangeInstantiationProcessor} - which is a \prettyref{feature:BeanInstantiationProcessor}). This reduces the complexity of the maintained code because here a correct implementation of PCEs is already provided in a generic manner. An example how to use the annotation and how to link a bean to PCEs of other PCE-capable beans can be found in the \javadoc{com.koch.ambeth.propertychange.PropertyChangeTest}{PropertyChangeTest}.

\begin{lstlisting}[style=Java,caption={Usage example for \textit{PropertyChangeAspect}}]
...
public class PCEListenerBean implements PropertyChangeListener
{
	@Override
	public void propertyChange(PropertyChangeEvent evt)
	{
		// do stuff with evt handle
	}
}
...
@PropertyChangeAspect
public abstract class PCECapableBean
{
	public static final String MY_PROPERTY_PROP_NAME = "MyProperty";

	public abstract String getMyProperty();

	public abstract void setMyProperty(String myProperty);
}
...
public void afterPropertiesSet(IBeanContextFactory beanContextFactory) throws Throwable
{
	IBeanConfiguration pceListenerBean = beanContextFactory.registerBean(PCEListenerBean.class);
	IBeanConfiguration pceCapableBean = beanContextFactory.registerBean(PCECapableBean.class);
	beanContextFactory.link(pceListenerBean).to(pceCapableBean, INotifyPropertyChanged.class);
}
\end{lstlisting}

Please note that in the example above the \type{PCECapableBean} is even an abstract class declaration. Of course this is just to show the depth of the implementation: In this case even the getter/setter pair including the correctly typed internal field is created by the bytecode enhancement module. Responsible for that enhancement is the \javadoc{com.koch.ambeth.bytecode.behavior.NotifyPropertyChangedBehavior}{NotifyPropertyChangedBehavior}.

\tip{Regarding the bytecode enhancement capability \AMBETH{} uses exactly the same implementation to enhance bean declarations here like it does to enhance entities for the \prettyref{module:Merge} and \prettyref{module:Cache} module. So it is also a basic feature for entities and is therefore designed to scale even with many thousands of objects - and many thousands of PCEs per object - very well.}