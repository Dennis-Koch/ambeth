\newcommand{\TODO}{{\textcolor[rgb]{1,0,0}{\Huge{TODO}}}}

\newcommand{\AMBETH}{{\emph\LARGE{Ambeth}}}

\newcommand{\type}[1]{\textit{#1}}

\newcommand{\typea}[2]{\textit{#2}}

\newcommand{\typeprop}[1]{\textit{#1}}

\newcommand{\prettyref}[1]
{\nameref{#1} (c.\ref{#1}, p.\pageref{#1})}

\newcommand{\shortprettyref}[1]
{c.\ref{#1}, p.\pageref{#1}}

\newcommand{\figureref}[1]
{f.\ref{#1}, p. \pageref{#1}}

\newcommand{\inputjava}[2]
{
	\lstinputlisting[caption={#1 (Java)},style=Java]
		{../../jambeth/#2}
}

\newcommand{\inputcsharp}[2]
{
	\lstinputlisting[caption={#1 (C\#)},style=Csharp]
		{../../ambeth/#2}
}

\newcommand{\javadoc}[2]
{\href{http://www.osthus.com/ambeth/javadoc/#1}{#2}}

\newcommand{\feature}[7]
{	
%\begin{wrapfigure}{r}{0.5\textwidth}
    %\begin{tabular}{ c | p{5cm}}
    %\hline
    %JavaDoc & \javadoc{#1}{#1} \\ \hline
		%Example & \javadoc{#7}{#7} \\ \hline
		%Environment & #2 \\ \hline
    %Since & #3 \\ \hline
		%Ticket & #4 \\ \hline
		%Config & #5 \\ \hline
		%Module & \prettyref{#6} \\ \hline
    %\end{tabular}
%\end{wrapfigure}
}

\newcommand{\tip}[1]
{
	\mdfdefinestyle{mystyle}
	{
		%shadow=true,
		%shadowsize=5pt,
		%shadowcolor=darkyellow,
		linewidth=5pt,
		%leftmargin=1cm,
		linecolor=yellow,
		roundcorner=10pt
		frametitlerule=true,
		frametitle={Did you know?}
	}
	\begin{mdframed}[style=mystyle]
		#1
	\end{mdframed}
}

\newcommand{\showimg}[1]
{
	\begin{figure}[!htbp]
		\centering
		\edef\tmp{\noexpand\includegraphics[width=0.75\linewidth]{\showimgref}}\tmp
		%\includegraphics[width=0.33\textwidth, angle=30]{#2}
		\caption{#1}
		\label{img:\showimgref}
	\end{figure}
}

\newcommand{\showimgfull}[1]
{
	\begin{figure}[!htbp]
		\centering
		\edef\tmp{\noexpand\includegraphics[width=\linewidth]{\showimgref}}\tmp
		%\includegraphics[width=0.33\textwidth, angle=30]{#2}
		\caption{#1}
		\label{img:\showimgref}
	\end{figure}
}

\newcolumntype{L}[1]{>{\raggedright\let\newline\\\arraybackslash\hspace{0pt}}m{#1}}
\newcolumntype{C}[1]{>{\centering\let\newline\\\arraybackslash\hspace{0pt}}m{#1}}
\newcolumntype{R}[1]{>{\raggedleft\let\newline\\\arraybackslash\hspace{0pt}}m{#1}}

\def\javalogo{img/Java-Logo}
\def\csharplogo{img/csharpdotnet}

\newcommand{\AvailableInJavaOnly}[1]
{	
	\AvailableInJavaAndCsharpInternal{#1}{\includesvg[\hsize]{\javalogo}}{}
}

\newcommand{\AvailableInCsharpOnly}[1]
{
	\AvailableInJavaAndCsharpInternal{#1}{}{\includegraphics[width=\hsize]{\csharplogo}}
}

\newcommand{\AvailableInJavaAndCsharp}[1]
{
	\AvailableInJavaAndCsharpInternal{#1}{\includesvg[\hsize]{\javalogo}}{\includegraphics[width=\hsize]{\csharplogo}}
}

\newcommand{\AvailableInJavaAndCsharpInternal}[3]
{
	\begin{tabular}{ L{0.77\linewidth} C{0.06\linewidth} C{0.15\linewidth} } #1 & #2 & #3 \\
	\end{tabular}	
}

%% Include SVG graphics
%% see http://laclaro.wordpress.com/2011/07/30/svg-vektorgrafiken-in-latex-dokumente-einbinden/
\newcommand{\executeiffilenewer}[3]{%
  \ifnum\pdfstrcmp{\pdffilemoddate{#1}}%
  {\pdffilemoddate{#2}}>0%
  {\immediate\write18{#3}}\fi%
}
% set inkscape binary path according to operating-system
\IfFileExists{/dev/null}{%
  \newcommand{\Inkscape}{inkscape }%
  }{%
  \newcommand{\Inkscape}{"../../../../inkscape/App/Inkscape/inkscape.exe" }%
}

% includesvg[width]{file} command
\newcommand{\includesvg}[2][1]{%
  \executeiffilenewer{#2.svg}{#2.pdf}{%
  \Inkscape -z -C --file="#2.svg" --export-pdf="#2.pdf" --export-latex
	}
	\resizebox{#1}{!}{\input{#2.pdf_tex}}%
}

\newcommand{\mixinTip}[2]
{
	Responsible for this functionality is the \javadoc{#1}{#2}.
	\tip{Most mixins applied at runtime via the \AMBETH{} Bytecode Pipeline can be fully debugged. Here for example by inserting break-points in the source of the #2-mixin if you are interested in details at runtime. This does work for C\# as well as Java in exactly the same way.}
}