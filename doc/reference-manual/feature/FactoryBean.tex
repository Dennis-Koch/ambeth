\SetAPI{J-C}
\section{FactoryBean}
\label{feature:FactoryBean}
\ClearAPI

This type of bean controls the instantiation of another bean and \AMBETH{} uses the factory bean to aquire an instance of the real bean and inject it where needed. The container controls only the life-cycle of the factory bean. The created bean is not processed by the container in any way. A Factory Bean is defined just by implementing the interface IFactoryBean. It is autowired to the interface or class of the beans it produces.

There are multiple scenarios where a factory bean can be used:

\begin{itemize}
	\item The life-cycle of a bean is controlled outside of the container, e.g. a database connection provided by an application server.
	\item Instances of the bean should not be shared. In this case the factory creates a new instance for every call to the \type{getObject()} method.
	\item The bean has a complex instantiation process and/or no parameter-less constructor.
\end{itemize}

