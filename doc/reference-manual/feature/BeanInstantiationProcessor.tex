\SetAPI{J-C}
\section{BeanInstantiationProcessor}
\label{feature:BeanInstantiationProcessor}
\ClearAPI
Instantiationprocessor beans are defined by implementing the interface \type{de.osthus.ambeth.ioc.IBeanInstantiationProcessor} and are used in order to customize the instantiation process of a bean. In most cases this is helpful if you want to dynamically subclass by bytecode enhancement the bean type to change or extend its behavior.\newline
The benefit of bytecode enhancement over proxying for AOP relies in the fact that it is very fast at invocation time (no reflection necessary, no CgLib interceptor or anything similar. As an outcome of the \textit{InstantiationProcessor} a single instance of a bytecode enhanced subclass of the target bean declaration is created.

\tip{The bytecode AOP approach to enhance beans does not create another instance of a potential abstract class with potentially unwanted behavior like an proxy-based AOP approach does - e.g. the abstract class might call foreign static methods. So with bytecode AOP it is not possible to circumvent any enhanced behavior whereas an AOP proxy can be circumvented with reflection and a direct interaction with the unmodified target bean can be done. As an additional drawback the AOP proxy approach has to handle cases where the method result of a target bean invocation is the target bean itself - should the proxy return the target bean then or return itself as a proxy again? Bytecode AOP does not have this issue.
\end{itemize}

 