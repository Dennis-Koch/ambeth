\section{Polymorphic Distance}
\label{feature:PolymorphicDistance}
For a given requested conversion the corresponding \emph{Polymorphic Distance} is calculated for all registered extensions compared to a registered extension as a scalar value. The extension with the smallest scalar value to the requested type will be selected to process the conversion. The calculation is done like this:
\newline
\begin{enumerate}
	\item Given object O of class C and given extension E registered for class CE
	\item If C is not inheriting (or equal) to CE => polymorphic distance is -1 (=NO\_VALID\_DISTANCE)
	\item If C is equal to CE => polymorphic distance is 0 (perfect match)
	\item If C is the type of Object itself => polymorphic distance is Integer.MAX\_VALUE
	\item If C is an array, calculate the distance for the component type CO of C instead of C itself => jump to step 2 in this algorithm and proceed
	\item For each interface I declared by C calculate its polymorphic distance to CE and (if there is a valid distance >= 0) add 10000 to its effective distance for comparison with others (because it is only an interface)
	\item If C has a super class SC, calculate the distance of SC to CE and add 1 to its effective distance for comparison with others
	\item Select the lowest (effective) distance between the results of C, SC and I as result of the calculation
\end{enumerate}

Examples:
\begin{itemize}
	\item \textit{java.util.ArrayList} <> \textit{java.util.ArrayList} => 0
	\begin{itemize}
		\item because the class ArrayList is equal to itself
	\end{itemize}
	\item \textit{java.util.ArrayList} <> \textit{java.util.LinkedList} => -1 NO\_VALID\_DISTANCE
	\begin{itemize}
		\item because LinkedList is not extended/inherited by ArrayList on any hierarchy level
	\end{itemize}
	\item \textit{java.util.ArrayList} <> \textit{java.util.List} => 10000
	\begin{itemize}
		\item because List is an interface and is explicitly implemented by ArrayList
	\end{itemize}
	\item \textit{java.util.LinkedList} <> \textit{java.util.List} => 10000
	\begin{itemize}
		\item because List is an interface and is explicitly implemented by LinkedList
	\end{itemize}
	\item \textit{java.util.LinkedList} <> \textit{java.util.Collection} => 10003
	\begin{itemize}
		\item 10000 because Collection is an interface and is explicitly implemented by \textit{AbstractCollection}, 3 because \textit{AbstractCollection} is inherited transitively across 3 levels by \textit{LinkedList} (\textit{LinkedList} > \textit{AbstractSequentialList} > \textit{AbstractList} > \textit{AbstractCollection})
		\item Note that there is also the potential result of 20000 with the argumentation that \textit{Collection} is implemented by \textit{List} (+10000) and \textit{List} is explicitly implemented by \textit{LinkedList} (+10000 again). So applying step 6 from the algorithm above 2 times is indeed true and is also internally considered - but the effective result will always be the lowest of all valid polymorphic distances. Leading to the fact that transitivity across super classes is preferred to transitivity across super interfaces.
	\end{itemize}
\end{itemize}