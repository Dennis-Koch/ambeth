%% feature-condition=SignatureUtil
\SetAPI{J}
\section{Cryptography}
\label{feature:Cryptography}
\ClearAPI
\TODO
\type{Ambeth} considers cryptographic security rigorously:

\begin{itemize}
	\item For password hashing \emph{intentionally slow} cryptographic hash algorithms are used (e.g. \emph{PBKDF2WithHmacSHA1} with several thousand iterations)
		\begin{itemize}
			\item As a result the single step during a user-login: check given password for correctness - may take 100ms or more. This is intentional and is nothing a user might complain in this scenario but essential to be practically immune to rainbow table or brute-force attacks.
			\item Configure with
				\begin{itemize}
					\item \prettyref{configuration:SecurityCryptoPaddedkeyAlgorithmName}
					\item \prettyref{configuration:SecurityCryptoKeyspecName}
				\end{itemize}
		\end{itemize}		
	\item Each generated or user-defined password has a \textit{cryptographic secure} generated long salt
		\begin{itemize}
			\item Further robustness against rainbow table or brute-force attacks: An attacker is not able to attack more than one secret at once - instead of being able to attack all encrypted secrets in parallel without a salt, a poor salt or only a single global salt for all secrets.
			\item Configure length of the salt with \prettyref{configuration:SecurityCryptoPaddedkeySaltsize}
		\end{itemize}	
	\item New salts are transparently generated whenever signatures or passwords are changed - for each changed secret individually.
	\item \emph{Optional:} Salts can be globally encrypted by the server admin which then makes it impossible to crack a secret without knowing the ``clear-text'' salt even if an attacker might get full access to the underlying database system.
		\begin{itemize}
			\item Configure with \prettyref{configuration:SecurityLoginSaltPassword}
		\end{itemize}
	\item Each algorithm configuration (one for each maintained secret) uses a \textit{cryptographic secure} generated IV (initialization vector) - instead of a default-zero-filled IV.
	\item During long-term productional usage of an \AMBETH{} installation the necessary security parameters can be adjusted according to potential cryptographic ``leak'' news or scientic breakthough in quantum computing
		\begin{itemize}
			\item Whenever a user logs in after a corresponding runtime-adjustment in the \AMBETH{} installation the provided cleartext password gets rehashed considering the new and more state-of-the-art security parameters.
			\item If the embedded \type{Ambeth Signature} feature is used a new public/private key pair will be generated transparently, newly encrypted with the provided cleartext password the user considering the new and more state-of-the-art security parameters
			\item Configure with
				\begin{itemize}
					\item \prettyref{configuration:SecurityLoginPasswordAutorehashActive}
					\item \prettyref{configuration:SecurityLoginPasswordAlgorithmIterationcount}
					\item \prettyref{configuration:SecurityLoginPasswordAlgorithmName}
					\item \prettyref{configuration:SecurityLoginSaltAlgorithmName}
					\item \prettyref{configuration:SecurityLoginSaltKeyspecName}
					\item \prettyref{configuration:SecurityLoginSaltLength}
				\end{itemize}	
		\end{itemize}	
\end{itemize}

If you are interested in the basic assumptions regarding password hashing please refer to \cite{lka01}, 

\tip{As with all mentioned cryptographic configurations in \AMBETH{} the server admin is able to set/update the global salt password without server restart. Of course all managed salts of all secrets will be transparently decrypted with the old password and encrypted with the new password on-the-fly in a transactional scope - blocking concurrent cryptographic process steps for a short time according to ACID constraints.}

\inputjava{A \type{ISignature} entity holds all information regarding public/private key functionality}
	{jambeth-security-server/src/main/java/de/osthus/ambeth/security/model/ISignature.java}

Of course the property \type{ISignature}.\typeprop{PrivateKey} holds \emph{encrypted} information at runtime: Only the signature-owning user can decrypt with his own password the private key of his signature (e.g. to sign an audited operation). Please refer to \prettyref{module:Audit} regarding handling of signatures to ease audit trails.

\inputjava{Maintainable cryptographic security with \type{IPBEConfiguration}}
	{jambeth-security-server/src/main/java/de/osthus/ambeth/security/IPBEConfiguration.java}

\begin{lstlisting}[style=SQL-Oracle,caption={Example schema definition to persist the \AMBETH{} \type{ISignature}-entity (Oracle SQL)}]
CREATE TABLE "SIGNATURE"
(	
	"ID"         			NUMBER(9,0) NOT NULL,
	"CREATED_ON" 			TIMESTAMP,
	"UPDATED_ON" 			TIMESTAMP,
	"UPDATED_BY" 			VARCHAR2(64 CHAR),
	"CREATED_BY" 			VARCHAR2(64 CHAR),
	"VERSION"    			NUMBER(9,0) NOT NULL,
	"USER_ID"				NUMBER(9,0),
	"SIGNATURE_ALGORITHM"	VARCHAR2(64 CHAR) NOT NULL,
	"KEY_FACTORY_ALGORITHM"	VARCHAR2(64 CHAR) NOT NULL,
	"ENCRYPTION_ALGORITHM"	VARCHAR2(64 CHAR) NOT NULL,
	"ENCRYPTION_KEY_SPEC"	VARCHAR2(64 CHAR) NOT NULL,
	"ENCRYPTION_KEY_IV"		VARCHAR2(173 CHAR) NOT NULL,
	"PADDED_KEY_ALGORITHM"	VARCHAR2(64 CHAR) NOT NULL,
	"PADDED_KEY_SIZE"		NUMBER(9,0) NOT NULL,
	"PADDED_KEY_ITERATIONS"	NUMBER(9,0) NOT NULL,
	"PADDED_KEY_SALT_SIZE"	NUMBER(9,0) NOT NULL,
	"PADDED_KEY_SALT"		VARCHAR2(64 CHAR) NOT NULL,
	"PRIVATE_KEY"			VARCHAR2(700 CHAR) NOT NULL,
	"PUBLIC_KEY"			VARCHAR2(700 CHAR) NOT NULL,
	CONSTRAINT "SIGNATURE_PK" PRIMARY KEY ("ID") USING INDEX,
	CONSTRAINT "SIGNATURE_FK_USER_ID" FOREIGN KEY ("USER_ID") REFERENCES "USER" ("ID") DEFERRABLE INITIALLY DEFERRED
);
\end{lstlisting}
In the examples above the properties of \type{IPBEConfiguration} are mapped as embedded properties of \type{ISignature} to the table ``SIGNATURE''. Technically it is also possible to shape a one-to-one relationship and persist \type{IPBEConfiguration} in an own table. The above definition is the recommended approach.\footnote{Please note the specific maximum length of the corresponding varchar-columns:\newline

As an example the length ``PUBLIC\_KEY'' it is intentionally defined to hold up to 700 characters. This is enough to store a binary array in base64-encoding up to a length of 523 bytes (roughly 75\% -2 bytes). 523 bytes are 4184 bits which are enough to hold e.g. an RSA key of 4096 bits. 4096 is at the current time of writing this (year 2014) considered a ``safe'' key length for RSA encryption for at least the next 30 years if you trust the US NSA\cite{nsa01}.}\newline



