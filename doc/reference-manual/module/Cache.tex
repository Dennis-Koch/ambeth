\SetAPI{J-C}
\section{Cache}
\label{module:Cache}
\ClearAPI
\TODO
%% MAVEN GENERATED START
\begin{lstlisting}[style=POM,caption={Maven modules to use \emph{Ambeth Cache}}]
<dependency>
	<groupId>de.osthus.ambeth</groupId>
	<artifactId>jambeth-cache</artifactId>
	<version>§\version§</version>
</dependency>

<dependency>
	<groupId>de.osthus.ambeth</groupId>
	<artifactId>jambeth-cache-bytecode</artifactId>
	<version>§\version§</version>
</dependency>

<dependency>
	<groupId>de.osthus.ambeth</groupId>
	<artifactId>jambeth-cache-datachange</artifactId>
	<version>§\version§</version>
</dependency>

<dependency>
	<groupId>de.osthus.ambeth</groupId>
	<artifactId>jambeth-cache-server</artifactId>
	<version>§\version§</version>
</dependency>

<dependency>
	<groupId>de.osthus.ambeth</groupId>
	<artifactId>jambeth-cache-stream</artifactId>
	<version>§\version§</version>
</dependency>
\end{lstlisting}
%% MAVEN END
\subsection{Features}
\begin{itemize}
	%% FEATURES START
	\item \prettyref{feature:CacheFactory}
	\item \prettyref{feature:CacheHierarchy}
	\item \prettyref{feature:CacheWalker}
	\item \prettyref{feature:EntityEquals}
	\item \prettyref{feature:ValueHolderContainer}
	%% FEATURES GENERATED START
	%%\item \prettyref{feature:CacheMapEntryBehavior}
	%%\item \prettyref{feature:DataObjectBehavior}
	%%\item \prettyref{feature:InterfaceImpl}
	%%\item \prettyref{feature:DefaultPropertiesBehavior}
	%%\item \prettyref{feature:EmbeddedTypeBehavior}
	%%\item \prettyref{feature:EnhancedTypeBehavior}
	%%\item \prettyref{feature:EntityEqualsBehavior}
	%%\item \prettyref{feature:InitializeEmbeddedMemberBehavior}
	%%\item \prettyref{feature:LazyRelationsBehavior}
	%%\item \prettyref{feature:NotifyPropertyChangedBehavior}
	%%\item \prettyref{feature:ParentCacheHardRefBehavior}
	%%\item \prettyref{feature:RootCacheValueBehavior}
	%% FEATURES END
\end{itemize}

%% CONFIGURATION GENERATED START
\subsection{Configuration}
\begin{itemize}
	\item \prettyref{configuration:CacheAsyncpropertychangeActive}
	\item \prettyref{configuration:CacheChildOnupdateOverwritetomany}
	\item \prettyref{configuration:CacheFirstlevelType}
	\item \prettyref{configuration:CacheFirstlevelWeakActive}
	\item \prettyref{configuration:CacheLruThreshold}
	\item \prettyref{configuration:CachePropertychangeFireoldvalueActive}
	\item \prettyref{configuration:CacheResultcacheActive}
	\item \prettyref{configuration:CacheSecondlevelActive}
	\item \prettyref{configuration:CacheSecondlevelWeakActive}
	\item \prettyref{configuration:CacheServiceName}
	\item \prettyref{configuration:CacheValueholderOnEmptyToOne}
	\item \prettyref{configuration:CacheWeakrefCleanupInterval}
	\item \prettyref{configuration:QueryBehavior}
\end{itemize}
%% CONFIGURATION END

\subsection{2nd level cache hierarchy}

\def\showimgref{img/gen/2013-01-02-DeK-RootCache-Request-Processing-1}
\showimg{A single autowired bean \type{IRootCache} represents the entry point to the whole 2nd level cache hierarchy (\textit{SLC})}

\begin{itemize}
	\item Without a current (thread-bound) transaction
		\begin{enumerate}
			\item Any call to the ``public'' \type{IRootCache} will be intercepted by the \type{ThreadLocalRootCacheInterceptor}.
			\item If no transaction is open the call is forwarded ``as-is'' to the committed RootCache instance (depending on the environmental configuration there will be a single shared instance (Mode A) or a thread-bound instance to represent the committed state of the database (Mode B).
			\item Any cache miss of the committed RootCache instance will be forwarded to the \type{CacheRetrieverRegistry} which respresents some kind of gateway where any type of information source can be connected to. \prettyref{extendable:ICacheRetrieverExtendable} shows how to use the corresponding extension points.
			\item For locally persisted entities (e.g. directly connected JDBC database) the \type{CacheRetrieverRegistry} forwards the request to the \type{DefaultPersistenceCacheRetriever} which is responsible to execute the necessary \emph{JDBC}-calls in a short-running, read-only, ad-hoc transaction.
			\item The result of the \emph{JDBC}-calls is stored in the committed RootCache for later use by any future thread (Mode A) or in the current thread only (Mode B).
		\end{enumerate}
	\item With an uncommitted (thread-bound) transaction
		\begin{enumerate}
			\item Any call to the ``public'' \type{IRootCache} will be intercepted by the \type{ThreadLocalRootCacheInterceptor}.
			\item If a transaction is open the call is forwarded ``as-is'' to the thread-bound transactional RootCache instance respresenting uncommitted data in the pending transaction
			\item Any cache miss of the transactional RootCache instance will be forwarded to the committed RootCache instance where the call itself is marked with \type{FailEarly} which prohibits the committed RootCache to populate any own cache misses by itself. This is important to prevent to ``pollute'' unintentionally the shared committed RootCache with uncommitted data of the thread-bound transaction.
			\item Any remaining cache miss will be forwarded to the \type{CacheRetrieverRegistry} which respresents some kind of gateway where any type of information source can be connected to. \prettyref{extendable:ICacheRetrieverExtendable} shows how to use the corresponding extension points.
			\item For locally persisted entities (e.g. directly connected JDBC database) the \type{CacheRetrieverRegistry} forwards the request to the \type{DefaultPersistenceCacheRetriever} which is responsible to execute the necessary \emph{JDBC}-calls in a short-running, read-only, ad-hoc transaction.
			\item The result of the \emph{JDBC}-calls is stored in the thread-bound transactional RootCache for later use in the current thread.
		\end{enumerate}
\end{itemize}
