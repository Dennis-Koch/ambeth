\SetAPI{J-C}
\section{Audit}
\label{module:Audit}
\ClearAPI
%% MAVEN GENERATED START
\begin{lstlisting}[style=POM,caption={Maven modules to use \emph{Ambeth Audit}}]
<dependency>
	<groupId>com.koch.ambeth</groupId>
	<artifactId>jambeth-audit</artifactId>
	<version>§\version§</version>
</dependency>

<dependency>
	<groupId>com.koch.ambeth</groupId>
	<artifactId>jambeth-audit-server</artifactId>
	<version>§\version§</version>
</dependency>
\end{lstlisting}
%% MAVEN END
\TODO
\subsection{Features}
\begin{itemize}
	%% FEATURES START
	\item \prettyref{annotation:Audited}
	\item \prettyref{feature:AuditedEntity}
	\item \prettyref{feature:AuditedService}
	%% FEATURES GENERATED START
	%% FEATURES END
\end{itemize}
\subsection{Usage}
To use the audit module, a few steps are necessary.
\begin{itemize}

	\item \prettyref{extendable:ITechnicalEntityTypeExtendable}: map the Ambeth internal entities like IAuditEntry to the technical, application model AuditEntity for example.


	
\end{itemize}

%% CONFIGURATION GENERATED START
\subsection{Configuration}
\begin{itemize}
	\item \prettyref{configuration:AuditActive}
	\item \prettyref{configuration:AuditEntityDefaultmodeActive}
	\item \prettyref{configuration:AuditEntityPropertyDefaultmodeActive}
	\item \prettyref{configuration:AuditHashalgorithmName}
	\item \prettyref{configuration:AuditProtocolVersion}
	\item \prettyref{configuration:AuditReasonRequiredDefault}
	\item \prettyref{configuration:AuditServiceDefaultmodeActive}
	\item \prettyref{configuration:AuditServiceargDefaultmodeActive}
	\item \prettyref{configuration:AuditVerifyCrontab}
	\item \prettyref{configuration:AuditVerifyMaxtransactionTime}
	\item \prettyref{configuration:AuditVerifyOnload}
\end{itemize}
%% CONFIGURATION END

\subsection{Activate Audit}
\textit{Ambeth Audit} is deactivated by default. To activate it set the following environmental flag:
\begin{lstlisting}[style=Props]
audit.active=true
\end{lstlisting}

Without any further customization step with activated \textit{Ambeth Audit} the following will happen at runtime:
\begin{itemize}
	\item All beans annotated with \type{\@Audited} on their class definition will have \emph{all} their public methods/services audited unless further customized - See \prettyref{feature:AuditedService}
	\item \emph{All} entities - even without a \type{@Audited} annotation on their class definition - will have \emph{all} their properties audited unless further customized. ``All properties'' does include all technical properties (PK,Version,...) as well - See \prettyref{feature:AuditedEntity}
\end{itemize}

This ``default-all-behavior'' can be changed by setting one or more of the following environmental flags:
\begin{lstlisting}[style=Props]
audit.entity.defaultmode.active=false
audit.entity.property.defaultmode.active=false
audit.service.defaultmode.active"=false
\end{lstlisting}

\subsection{Audit Information Model}

\def\showimgref{img/gen/2014-10-04-DeK-Ambeth-Audit-3}
\showimg{Audit Information Model}

The \type{AuditEntry}.\typeprop{UserIdentifier} is retrieved from the current \type{User} entity while processing an audited operation. The sign algorithm itself is fully configured by the signature specification which may be specific for each user.

\tip{The \type{Audit Information Model} is able to contain not only changes to primitive properties but also relational changes in any possible scenario as well. An additional sophisticated behavior can be seen in the fact that if you configure \type{Ambeth Audit} to cover all properties of all entities invariably you are incidentally blessed with a gapless history of the whole information model of your application: Considering the necessary \type{AuditEntry} instances it is conceptionally possible to ``recover'' the state of your full information model to any point of time in history - not only of a single property of a specific entity instance.}

\def\showimgref{img/gen/2014-10-04-DeK-Ambeth-Audit-4}
\showimg{User Information Model}

For more information about the signature configuration \& algorithms please take a look at \prettyref{feature:Cryptography}.

\subsection{Verify existing AuditEntry manually}
Any number of instances of an \type{AuditEntry} can be verified using the \type{IAuditEntryVerifier}-Bean:

\inputjava{The \type{IAuditEntryVerifier}-Bean}
	{jambeth-audit/src/main/java/com/koch/ambeth/audit/IAuditEntryVerifier.java}

The internal verifying algorithm uses the signature configuration and the public key of the signature owned by the user which did the signing of each \type{AuditEntry}.

\subsection{Verify AuditEntries automatically}
It is also possible to verify all audited entities that are loaded by ambeth. If the business requirement expects that every loaded entity is verified it is possible to enable this behavior. Attention this will produce a higher load in the application because every entity that is loaded will be verified prior to be available in the \type{IRootCache}. 

\TODO add links to the configuration options.
 
To explain the algorithm in more detail, a short, not complete ex course to the audit is needed.
How does Ambeth stores audit information?
\pagebreak
\def\showimgref{img/audit-verify.png}
\showimg{Basic, not complete audit structure.}

\begin{itemize}
	\item IAuditEntry: This entity encapsulates one audit operation. E.g. one database transaction with multiple entities.
	\begin{itemize}
		\item Signature=char[]: The combined cryptographically hash for one audit entry. It is based on the AuditedEntities Signature.
	\end{itemize}
	\item IAuditedService: This is used to audit service calls. It is out of scope for the entity verify.
	\item IAuditedEntity: One audit entry can have multiple entities (all entities from one transaction for example)
	\begin{itemize}
		\item ChangeType: was the AuditedEntity: created, updated or deleted
	\end{itemize}
	\item IAuditedEntityPrimitiveProperty: Only the modified properties will be stored in the IAuditedEntity object.
	\begin{itemize}
		\item Name of the primitive property that was changed
		\item NewValue of the primitive property. To get the ``OldValue'' a lookup back in the audit history is necessary.
	\end{itemize}
	\item IAuditedEntityRelationPropertyItem: Also relations and changes on the relations are audited
	\begin{itemize}
		\item ChangeType: was the relation added or removed from the audited entity
		\item Ref: the reference to the target object of the relation
	\end{itemize}
	\item Signature = char[]: The signature for the current IAuditedEntity. Every IAuditedEntity gets its own signature. All IAudtedEntity signatures together will be combined in the IAuditEntry.Signature.
\end{itemize}
\TODO verify the following with the actual implementation.
The audit verify algorithm does work in the following sequence:
\begin{enumerate}
	\item The application requests an entity. e.g. ORI(Entity:Project, ID:3, Version:1), which is not in the root cache by now.
	\item The cache retriever (see \prettyref{extendable:ICacheRetrieverExtendable} ) loads this entity from the persistence layer.
	\item ... \TODO ... magic happens here...
	\item The loaded entity which needs to be verified is the ``current state'' 
	\item Ambeth now loads the corresponding ``to be state'' from the audit trail.
	\item Because we only need the ``current to be state'' we can load the audit trail backwards, starting with the newest entry for the matching ORI.
	\item Tip 1: As mentioned earlier, only the ``changes'' on entities are audited
	\item Tip 2: Ambeth is aware of all properties and relations of an entity
	\item Those to tip's in mind, Ambeth loads IAuditedEntries with IAuditedEntities (matching the ORI) as long as all properties for the ``to be state'' are present.
	\item While loading the IAuditedEntities, every IAuditedEntity's signature is verified against the users key.
	\item If Ambeth collected all properties for the ``to be state'' it can be compared to the ``current state'' in the database. If the check fails, an error message is logged to the application log.
\end{enumerate}

\subsection{Customize Entity Audit}
Despite this annotation-based approach to configure the audited information mentioned in \prettyref{feature:AuditedEntity} it is possible to fully customize what changes on an entity should be audited by any kind of business rule at runtime. Just link your configuration to the \prettyref{extendable:IAuditConfigurationExtendable} during container start. If you intend to register your configuration after the container start be aware that your configuration needs to be registered before this configuration is needed for the first time (e.g. when the corresponding entity gets changed \& persisted).

\subsection{Customize audit serialization protocol}
During evolvement of an application it may be necessary to even change the serialization approach of the audit process of \type{Ambeth}. With a naive approach all existing \type{AuditEntry} instances in the persistence layer could not be verified any more if another protocol is used. So the chosen approach is to define a protocol version and link your custom serialiation protocol with your newly defined protocol version. The default shipped protocol version is ``1'' and is implemented in \type{AuditEntryWriterV1}. It is recommended to start your custom versioning with at least number ``10'' to be compatible with future evolvement of Ambeth.

\inputjava{\TODO}
	{jambeth-audit-server/src/main/java/com/koch/ambeth/audit/server/IAuditEntryWriter.java}
		
\subsection{Managing the signature of a user}
\TODO

\subsection{AuditEntry}
\TODO

\subsection{Audit information over time}
\TODO

\subsection{Audit information over time by different users}
\TODO

\subsection{Retrieving a state of an entity in audited history}
\TODO

\subsection{Covered audited information}
\TODO