\SetAPI{J-C}
\section{DataChange}
\label{module:DataChange}
\ClearAPI
\TODO
%% MAVEN GENERATED START
%% MAVEN END
\subsection{Features}
\begin{itemize}
	%% FEATURES START
	\item \prettyref{feature:DataChange}
	%% FEATURES GENERATED START
	%% FEATURES END
\end{itemize}

%% CONFIGURATION GENERATED START
\subsection{Configuration}
\begin{itemize}
	\item \prettyref{configuration:DatachangePersistenceKeepeventsMillis}
\end{itemize}
%% CONFIGURATION END