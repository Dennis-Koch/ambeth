\SetAPI{J-C}
\section{CompositeIdBehavior}
\label{feature:CompositeIdBehavior}
\ClearAPI
Allows to create objects at runtime where their equals() hashcode() toString() methods behave like a composite id (unique identifier). This way it is possible for \AMBETH{} to pass composite id information via the Member-API as a single argument / response. At runtime the created object has public fields which are named like the mapped primitive members forming the composite id (unique identifier)

\begin{lstlisting}[style=Java,caption={Example usage to receive an instance of a composite id (Java)}]
@Autowired
protected ICompositeIdFactory compositeIdFactory;

@Autowired
protected IEntityMetaDataProvider entityMetaDataProvider;

public Object extractCompositeId(Object myObject)
{
	IEntityMetaData metaData = entityMetaDataProvider.getMetaData(myObject.getClass());
	return compositeIdFactory.createCompositeId(metaData, metaData.getIdMember(), myObject.getId1(), myObject.getId2());
}
\end{lstlisting}
\begin{lstlisting}[style=Csharp,caption={Example usage to receive an instance of a composite id (C\#)}]
[Autowired]
public ICompositeIdFactory CompositeIdFactory { protected get; set; }

[Autowired]
public IEntityMetaDataProvider EntityMetaDataProvider { protected get; set; }

public Object ExtractCompositeId(Object myObject)
{
	IEntityMetaData metaData = EntityMetaDataProvider.GetMetaData(myObject.GetType());
	return CompositeIdFactory.CreateCompositeId(metaData, metaData.IdMember, myObject.Id1, myObject.Id2);
}
\end{lstlisting}
