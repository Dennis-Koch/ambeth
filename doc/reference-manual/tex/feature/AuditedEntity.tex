\SetAPI{J-C}
\section{Audited Entity}
\label{feature:AuditedEntity}
\ClearAPI
\TODO
\feature
	{com.koch.ambeth.audit.Audited}
	{Java}
	{2.1.50}
	{-}
	{-}
	{module:Audit}
	{-}

\inputjava{Usage example for the \type{Audited}-Annotation on an entity}
	{jambeth-examples/src/main/java/com/koch/ambeth/example/audit/AuditEntityUsageExample.java}

The example above annotates an entity to audit some of its properties. ``Audit'' means here that there will be a corresponding entry in the information model:

\begin{itemize}
	\item If not already exists: A new instance of \type{com.koch.ambeth.audit.AuditEntry} for the current thread-bound transaction. At the point where \type{Ambeth} does evaluate the audited information there is always a transaction bound.
	\item A new instance of \type{com.koch.ambeth.audit.AuditedEntity} for each changed audited entity found during the merge process. Each \type{AuditedEntity} is associated to the previously mentioned \type{AuditEntry}
\end{itemize}

For further information how to customize which information to audit at runtime please take a look at \prettyref{module:Audit}.


 